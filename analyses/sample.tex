\documentclass[D:/studies/WinnerS/Erhebungen/IPhO1718/paper/probsol_paper/main/TaylorFrancis/interactapasample]{subfiles}

%\usepackage{Sweave}

\begin{document}



The sample comprises students who participated in the online questionnaire and also submitted answers to the physics problem solving abilities items. 144 (m=101, f=43) students responded to these items. Mean (SD) age for the participants was 16.2 (1.2). In order to rule out potential selection biases with regards to the entire population of participants in the online questionnaire, background variables (gender and age) and affective/motivational variables were used to predict group membership through a logistic regression with group as outcome variable. Significant coefficients for a predictor variable would suggest that a certain predictor distinguishes the two groups. No effect became significant (Self-efficacy physics: $\beta=-0.08, SE(\beta)=0.12, z=-0.63, p=.53$; Expectancy physics competition: $\beta=0.09, SE(\beta)=0.13, z=0.67, p=.50$; Value physics competition: $\beta=0.03, SE(\beta)=0.12, z=0.3, p=.76$; Sense of belonging: $\beta=-0.02, SE(\beta)=0.12, z=-0.15, p=.88$; Social support in physics: $\beta=0.01, SE(\beta)=0.11, z=0.1, p=.92$; Qualified round: : $\beta=-0.21, SE(\beta)=0.15, z=-1.36, p=.17$; Gender: $\beta=0.11, SE(\beta)=0.25, z=0.46, p=.65$; Age: $\beta=0.01, SE(\beta)=0.1, z=0.12, p=.90$).


\end{document}
