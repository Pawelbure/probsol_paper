\documentclass[D:/studies/WinnerS/Erhebungen/IPhO1718/paper/problem_solving/main/TaylorFrancis/interactapasample]{subfiles}

\begin{document}

\section{Predictors for success in science Olympiads}

Olympians engage in a certain domain for an extended period of time. Expertise research is adamant in emphasizing that experts in various domains deliberately exercise for years or even decades \citep{Simon.1983}. Certainly, expertise in a domain is not the product of in-the-moment decisions, but rather continuously engaging in certain academic settings. Consequently, in order to characterize successful olympians in an enrichment program, a theory for long-term aspirations and academic choices is in need. One such framework was forwarded by \cite{Eccles.1983}, namely the expectancy-value model for achievement motivation. In this framework, relevant variables and connections amongst these variables to characterize successful students in the science Olympiads are outlined \cite{Eccles.1983,Urhahne.2012}. A parsimonious variant of the expectancy-value model is presented by \cite{Urhahne.2012} that was previously employed to characterize successful students in the Biology and Chemistry Olympiads. The model assumes, broadly summarized, that students engage in educational environments (e.g., tasks, choices, enrichment programs) on the basis of expectancies and values that they have towards these environments. Students who expect that they can achieve or be successful in certain environments accordingly make their academic choices in alignment with these expectations. Similarly, personal values towards the environment inform the academic choices. \cite{Eccles.1983} distinguished interest (personal enjoyment of subject), cost (resources required to excel in subject), utility (future usefulness of subject in personal life), and attainment (fit to own self) values that moderate academic choices. 

\begin{figure}
\centering
\subfile{D:/studies/WinnerS/Erhebungen/IPhO1718/paper/problem_solving/theory/fig/expectancy_value}
\caption{Motivational variables and cognitive variables predict success in Olympiads.}
\label{Fig:model}
\end{figure}


Two building blocks are assumed to characterize successful students in the programs: cognitive variables (e.g., problem solving skills, subject matter knowledge) and motivational variables (e.g., self-concept, sense of belonging, and motivation). These two building blocks will be expanded in the following:

 % Problem solving is ''any goal-directed sequence of cognitive operations'' \citep[][p. 257]{Anderson.1980} that proceeds from an initial state in a problem to a desired goal state, knowing or not knowing the means, and knowing or not knowing the goals \citep[e.g.,][]{Mayer.2013}. Problem solving is thus constituted by many facets (such as identifying important concepts/principles, applying strategies and mathematics). Some doubts were cast on whether problem solving is at all a cognitive phenomenon \citep[][]{Ericsson.2003}, however, physicists use problem solving all the time in their careers, and it has been found to range amongst to most used skills in later physics \citep[e.g.,][]{Pold.2016}.


\textit{Cognitive variables:} It is widely recognized that expertise is domain-specific and manifests itself in specific cognitive abilities unique to a certain domain \citep{McClelland.1973,Gardner.1993}. Different measures were employed to measure cognitive abilities. Cognitive variables comprise a vast range of skills such as reasoning skills. However, most pertinent to physics are domain specific conceptual knowledge and problem solving skills. In particular, physics problem solving is a broad set of subskills that comprise factual knowledge (declarative) and strategic (procedural) knowledge \citep{Anderson.1996}. This knowledge is organized in chunks (i.e., effective productions like schematas), and which is situation sensitive and conditionalized for experts. Hence, experts in physics have fast and frugal heuristics in order to comprehend physics problems appropriately. Dreyfuß quipped that experts do not solve problems but rather do what they usually do when facing problems. The conditionalized and chunked knowledge allows physics experts to categorize physics problems based on the relevant principles that are necessary for solving the problems (deep structure), whereas novices categorize problems by surface features such as ''is a pulley involved?'' \citep{Chi.1981}. Deep structure usually is what can be stated verbally in a principle, or in a formula \citep[p. 70]{Schwartz.2015}. It is argued that experts in semantically rich domains (e.g., engineering thermodynamics) store a vast amount of domain-specific conceptual knowledge also \citep{Bhaskar.1977} that they can match to the correct problems \citep{Chi.1981}. Experts show a careful qualitative problem analysis, and solve the problem based on the fundamental principles involved \citep{Larkin.1980,Chi.1981}. Experts tend to proceed in a logically well-posed sequence of steps of re-translations of the problem, until a certain line of argumentation is reached. They might intersperse a simple qualitative diagram in their responses to infer further steps \citep{Larkin.1980,Larkin.1987}. \cite{Fortus.2009} demonstrates that experts more readily find assumptions for ill-defined problems in order to solve these problems. In fact, experts' knowledge has been attributed to be ''conditionalized'' -- it includes a specification of the contexts in which it might be applied \citep{Simon.1980,Glaser.1992}. Novices have problems to recognize and reflect upon the assumptions that predicate the solution to a problem. \cite{Reif.1992} report that novice students showed deficient applicability conditions for concepts and thus the use of the concepts was flawed.

%They appear to rely on their conceptual knowledge (i.e., principles), before bringing in mathematics for problem solving while novices rely more on a piecemeal approach to problem solving \citep{Reif.2008,Jong.1986,Jong.1996}. Inserting mathematical equations early in the problem solving process was detrimental to solving the problem and is considered a poor strategy \cite{Ogilvie.2009,Walsh.2007}. Experts will utilize mathematics (note that mathematics is used here to imply formulas that are typical for physical principles, e.g., $F=ma$) only in the stage after they have properly figured out the physics that is necessary to solve the problem. This shortcut with mathematics was identified in multiple studies and with similar terms: \cite{Walsh.2009} identified an ''unstructured plug-and-chug'' heuristic to solve the physics problems, \cite{Tuminaro.2007} analyzed epistemic games of students and found the ''recursive plug-and-chug'' game, where students identified a target quantity and sought a suitable equation in order to calculate the target quantity, and \cite{Jong.1996} identified a strategy for physics problem solving which they called ''kick-and-rush.'' This strategy involves the following steps: find formula, plug in, calculate, ready. These approaches are constitutive for novice-like problem solving in physics \citep[see also:][]{vanHeuvelen.1991,Heller.1992}.

Other cognitive variables that potentially predict success in programs such as the science Olympiads are general cognitive abilities. A simplified version of Carroll's hierarchical theory of cognitive abilities is the radex model where quantitative/numerical, spatial/pictorial, and verbal/linguistic abilities are differentiated \citep{Snow.1997}. We will focus our discussion on the quantitative and numerical abilities, though we acknowledge that the other dimensions are equally important \citep{Wai.2009}. It can be hunched that cognitive abilities are important for success in the Olympiads since they tax problem solving abilities. Problem solving abilities, on the other hand, are related with intelligence. The correlations of intelligence and complex problem solving seem to depend largely on the utilized constructs (correlations range from $.00$ to about $.80$), it seems that particularly well-defined problems (i.e., specific goal-structure) correlate high with facets of intelligence \citep[summarized in:][]{Funke.2007}. Students that excel in STEM show generally high cognitive abilities. For example, \cite{Wai.2009} find that less than $10$ percent of those holding a STEM-PhD were below the top quartile in cognitive ability (comprising spatial visualization in 2D and 3D, mechanical reasoning, and abstract reasoning) during adolescence, and their data supports the conclusion that the importance of spatial ability for STEM increases with more advanced degrees (bachelor, master, PhD). Graduates in physical sciences such as engineering range among the top performers in cognitive abilities \citep[see also:][]{Shea.2001}. ''STEM disciplines place a premium on nonverbal ideation indicative of quantitative and spatial reasoning'' \citep{Lubinski.2010}. 

%However, an interesting phenomenon can be observed towards more expert students. Namely, experts in a STEM field tend to differentiate less by generic abilities \citep{Uttal.2012}. This conincides with the general observation that experts rely more on crystallized intelligence (i.e., intelligence that relies on prior knowledge) as compared to fluid intelligence (i.e., intelligence that requires no prior knowledge for problem solving) \citep{Sternberg.2003}. 

\textit{Motivational variables:} A more self-reflective host of predictor variables comprise motivational constructs. In particular, the expectancy of success towards enrichment programs was found to be a relevant factor for predicting success in enrichment programs \citep{Urhahne.2012}. In general, the expectation to excel in certain tasks relevant for a domain is an integral part of the achievement-motivation model \citep{Eccles.1983}. A closely related construct is the self-efficacy in physics (i.e., the perception to be able to solve physics problems). Self-efficacy is generally a good predictor for school success \citep[e.g.,][]{Britner.2006}. The expectancy-value model of achievement motivation predicts also that values towards tasks in the domain are relevant for engagement. However, these effect do only show up partially (for utility value) in reality \citep{Urhahne.2012}. Of yet unclear importance is the sense of belonging to the domain. It is documented that sense of belonging (the perceived belonging to a domain) is important for students to build the intention to study STEM \citep{Good.2012}, however, it is unclear if sense of belonging can explain further variance between students who engage in an Olympiad, i.e., students who already identify with the domain.

%Relatively recently, problem solving research engaged in the study of interindividual differences \cite{Davidson.2003}. \cite[p. 286]{Jay.1997} write: ''Abilities, knowledge, and strategies enable a person to problem find, and contexts provide the stimulus, but it is dispositions that actually promote the initiation of problem finding.'' Some researchers found personality traits that were associated with creativity, such as self-confidence and ambitiousness \cite{Feist.1998}, and tolerance of ambiguity \cite{Sternberg.1995}. However, usually great variability is found among creative individuals \cite{Davidson.2003}. Motivation was also found to be an important factor in the creative process. \citeA{Amabile.1996} noted curiosity, and a playful attitude amongst creative indiviudals.  ''Students think harder and process material more deeply when they are interested and believe that they are able to solve the problem (i.e., have high self-efficacy), according to Mayer's (1998) effort-based learning principle.'' (Jonassen, 2000, p. 71)

Cognitive and motivational varibales might predict success in the Physics Olympiad. But, what does success mean in this context? Success in the science Olympiads can be operationalized through the highest reached round for a participant. The Olympiads comprise several subsequent stages that require increasingly advanced knowledge in order to solve the posed problems \citep{Koehler.2017}. In order to characterize successful students with regards to cognitive and motivational variables, a host of variables was employed in the context of the Physics Olympiad. 

\begin{itemize}
\item To what extent is students' success in the Physics Olympiad a function of cognitive and motivational variables that are gleaned from the expectancy-value model of achievement motivation?
\end{itemize}

\end{document}