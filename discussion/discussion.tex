\documentclass[D:/studies/WinnerS/Erhebungen/IPhO1718/paper/problem_solving/main/TaylorFrancis/interactapasample]{subfiles}

\begin{document}

\section{Discussion}

Characterizing successful students in the Physics Olympiad was the research interest for this study. This research interest stems from the fact that too little is known about the characteristics of successful students who partake in enrichment programs such as the science Olympiads. Insights into characteristics of successful students might then advance our knowledge on expertise in domains such as physics and how to best design training materials and programs to foster expertise. The characterization utilized motivational and cognitive constructs that are gleaned from the expectancy-value model of achievement motivation and from expertise research. Motivational variables include, amongst others, expectancy of success, values towards the competition, sense of belonging, and social support for physics engagement. Cognitive variables include measures that were not all readily available from previous research. In particular, expertise research suggests that experts outperform novices in problem solving. Consequently, a physics problem measure was developed on the basis of relevant theory and prior measures for problem solving and general cognitive abilities were utilized. The developed measure for physics problem conceptualization ability shows satisfying item characteristics with regards to internal consistency as measured through Cronbach's $\alpha$, discriminatory power and interrater reliability. Furthermore, discriminatory validity as measured through correlation with similar constructs \citep{Campbell.1959} was satisfactory because the intercorrelations can be considered small to medium size. 

As a means to answer the research question and thus characterize successful students in the Physics Olympiad, an ordinal regression was used with a hierarchical inclusion procedure of sets predictor variables \citep{Field.2012}. First, the cognitive variables were included in the model because on the basis of prior research they were considered most important. In a second step, the cognitive and the motivational variables were included together in the model. Problem conceptualization ability was found to be significantly related with success in the Physics Olympiad, as assessed through highest qualified round. Besides problem conceptualization ability, also heuristic physics problem solving proved to be significantly related with success in the Physics Olympiad. Both, problem conceptualization ability and heuristic physics problem solving ability remained significantly related with success when the motivational variables were added into the model. Additionally, expectancy of success was significantly related to success in the Physics Olympiad. This is consistent with the findings from the Chemistry Olympiad \citep{Urhahne.2012}, and with more general findings related the expectancy of success \citep{Eccles.1995}. No other variables from the set proved to be a significant predictor.

The fact that the scope of this study was limited to the expectancy-value model of achievement motivation and expertise research is a limitation because other variables (e.g., possible science self, science peer relations) might be significanlty related with success in the Physics Olympiad and thus be characteristic of successful students. Further constraints in interpretability and generalizability arise with respect to selection bias. It should be noted that only a fraction of the students that participated in the Physics Olympiad also took part in the questionnaire. The here analyzed sample differed from the overall population with respect to achievement (they performed better), and with regards to age (they were younger). Two other important predictors, namely prior experiences in the competition and achievement in round 1, were omitted from the analyses. The rationale for this is that prior experience and achievement in the competition would cannibalize with other constructs that are more informative to the research interest. Namely, prior experiences can be considered another form to measure physics problem solving expertise. However, the interest in this study was to assess the relationship of problem solving expertise with success. Since neither prior experience nor achievement in the competition are measures that would inform training policies and practices in the context of these programs they were omitted.


\section{Conclusions}

The results from this study are of interest for educators that deal with expert students in the context of enrichment programs or in regular schools. In this study, success in these programs is significantly related to physics problem solving skills (problem conceptualization and heuristic physics problem solving), but not to generic cognitive abilities. However, physics problem solving expertise is malleable and thus open to intervention and training. Hence, programs that facilitate engagement for students in these programs are well advised to include elements of problem solving training and meta-cognitive skills on self-knowledge about problem solving. The highest achieving students in the problem conceptualization test presented highly intelligible and coherent accounts about their approaches to solve these problems. This included conceptual knowledge and also knowledge about applicability of concepts. Thus, these elements should be included in problem solving training -- namely outlining an explicit rationale about why something is used in a certain problem context \citep{Fortus.2009}. Postponing the mathematics in physics and simply reflecting on the assumptions for various problems and the necessary concepts might pay off in order to develop the highly valued qualitative problem solving approach in physics that is reflective of expertise. Furthermore, writing exercises that enable students to reflect their own problem solving were found to be effective. In the literature on meta-cognitive skills and problem solving interventions positive effects for the explicit reflection of strategies were found to be effective \citep[e.g.,][]{Bransford.1986,Mason.2010,Perels.2005}. Some of the students in this study simply wrote that they have not yet learned the contents that seemed to be required in the problems. Even though this is likely to be true, the answer is unfortunate, because modeling physical phenomena and writing about how to tackle the problem is possible without the specific content knowledge to a large extent. This would require schools to teach problem solving as a generic skills that is applicable in any situation.

Furthermore, as repeatedly demonstrated by psychological research, a positive expectation to be successful in the Physics Olympiad (e.g., expectancy of success) proved once again as a reliable predictor for success. It is not easily conceivable how students should develop personal feelings of competence and efficacy (and thus an expectation of success) towards the Physics Olympiad in particular. The interaction with this program oftentimes reduces to mere homework assignments and feedback from teachers. An intrinsic drive for competence and interest need be present in order for the students to muster feelings of competence from involvement in these programs. Potential measures that foster students' engagement in the Physics Olympiad are well advised to carefully consider how feelings of personal competence and efficacy can be made part of the experiences students make in such measures. Care need to be taken for student-student interactions. For example, it has been demonstrated that contexts were high-achieving students come together and solve problems are environments that also likely hamper personal feelings of competence and efficacy for these same students \citep{Marsh.1995}. 


\end{document}
