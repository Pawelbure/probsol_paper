\documentclass[../main/TaylorFrancis/interactapasample]{subfiles}

\begin{document}

\section{Introduction}

%Modern societies have been characterized to be increasingly knowledge based and technological \citep{Friedman.2005,Pinker.2018,Flynn.2018}, where the job market depends on specialized skills in science, technology, engineering, and mathematics (STEM) \citep{Wieman.2007,Rosling.2018}. However, an aging workforce in STEM and a growing job market with fewer students entering occupations in STEM pose threats to the competitive edge of modern societies for tackling global problems such as climate change, outsourcing of fossil fuels, and energy transformation towards renewabls \citep{Wieman.2007,Osborne.2003}. It is consequently a primary goal for educational systems to foster interests and skills for students towards STEM. Identifying and promoting interested students in STEM is part and parcel for sustenance of future wealth \citep{PCAST.2012}. 

Enrichment programs such as the ScienceOlympiads are means to identify and promote talented students in science, technology, engineering, and math (STEM). These programs proceed in subsequential stages where students compete with one each other \citep{Petersen.2017}. Ultimately, a national team is chosen, comprising about five students, who compete on an international level against students from more than couple dozen countries. Federal government and the STEM community endorse these means as viable instruments to foster talented students \citep{KMK.2009,Petersen.2017}--and educational researchers in gifted programs motivated the necessity of ScienceOlympiads (and programs alike) as a viable complementary to regular schools that have limited capacities to provide resources for talented students \citep{Reis.2010}. Besides these goals, research is scare on these programs and less elaborated are models that outline the mechanisms how these programs are meant to identify and promote talented or high-achieving students \citep{Ziegler.2004}.

Research suggests that enrichment programs such as the ScienceOlympiads can be effective in training facets of STEM skills and in developing interest and motivation of students towards STEM \citep{Oswald.2004,Aljughaiman.2012,Wai.2010,Marsh.1995}. Particularly, successful candidates report a positive impact on their future job aspirations in STEM through such programs \citep{Feng.2001,Oswald.2004,Subotnik.1993}. Gender differences amongst the participants are large: boys compared to girls receive more social support for their participation \citep[e.g.,][]{Lengfelder.2002,Urhahne.2012}. \cite{Urhahne.2012} found for the ChemistryOlympiad that previous participation was the best predictor for success in this competition and also expectancy of success distinguished successful participants from less successful participants \citep[similar findings in:][]{Stang.2014}.

More sobering were meta-analytic findings 

Models for talent development and achievement motivation commonly differentiate cognitive and affective variables that explain academic behavior (e.g., choices) for talented and high-achieving students \citep{Heller.2002,Ziegler.2004}.

For example, \cite{Urhahne.2012} found no appropriate measures in order to assess subject matter knowledge for these students and relate this knowledge to success in these programs. This is unfortunate since cognitive skills such as problem solving might comprise amongst the most tangible resources for students' success in science Olympiads \citep{Bransford.2000,Wai.2009}. 


Consequently, in this study we are particularly interested in characterizing successful students in a science Olympiad with regards to cognitive and motivational constructs in order to advance our understanding what constitutes successful participation in these programs. We capitalize on the physics Olympiad where we have a comprehensive literature base delineating the characteristic features of expertise. This enabled us to utilize research-based instruments to measure expertise in the Physics Olympiad from a cognitive vantage point and triangulate these results with more motivational measures that have been proposed elsewhere. In order to assess characteristics for success in the Physics Olympiad participating students were tracked throughout their engagement in the Physics Olympiad with a questionnaire. 

\end{document}
