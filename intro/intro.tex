\documentclass[D:/studies/WinnerS/Erhebungen/IPhO1718/paper/problem_solving/main/TaylorFrancis/interactapasample]{subfiles}

\begin{document}

\section{Introduction}

Modern societies have been characterized to be increasingly knowledge based and technological \citep{Friedman.2005}. Specialized analytic and scientific thinking skills are increasingly important in order to maintain health, wealth, and progress \citep[e.g.,][]{Pinker.2018,Flynn.2018}. These skills empower humans to understand the world and eventually tackle global problems such as climate change, outsourcing of fossil fuels, world hunger, etc. Consequently, literacy in science, technology, engineering, and mathematics (STEM) range amongst the most required goals to cope with future problems \citep{Wieman.2007,Rosling.2018}. However, literacy in STEM is threatened for multiple reasons such as an aging workforce and a shortage of skilled workers. We submit that identifying and promoting interested students in STEM is part and parcel for sustenance of future wealth \citep{PCAST.2012}.

As an elaborate means to identify and promote talented students in STEM, enrichment programs such as the science Olympiads were implemented \citep[e.g.,][]{Petersen.2017}. In science Olympiads highly interested students in the respective countries solve challenging science problems and eventually advance to an international competition where students from all over the world come together. Some studies suggest that such enrichment programs can be effective in training facets of STEM skills and in developing interest and motivation of students towards STEM \citep{Oswald.2004,Aljughaiman.2012,Wai.2010,Marsh.1995}. Particularly, successful candidates report a positive impact on their future job aspirations in STEM through such programs \citep{Feng.2001,Oswald.2004,Subotnik.1993}. However, it is less clear what particular characteristics successful candidates bring to the competitions. Gender differences amongst the participants in the science Olympiads such as more social support for boys compared to girls have been attributed to cause young women to underperform in particular \citep[e.g.,][]{Lengfelder.2002,Urhahne.2012}. \cite{Urhahne.2012} found for the Chemistry Olympiad that previous participation was the best predictor for success in this competition. Besides these more superficial features, the empirical basis for delinating characteristics of successful candidates is scarce, for once because appropriate measurement instruments are missing. For example, \cite{Urhahne.2012} found no appropriate measures in order to assess subject matter knowledge for these students and relate this knowledge to success in these programs. This is unfortunate since cognitive skills such as problem solving might comprise amongst the most tangible resources for students' success in science Olympiads \citep{Bransford.2000,Wai.2009}. 

Consequently, in this study we are particularly interested in characterizing successful students in a science Olympiad with regards to cognitive and motivational constructs in order to advance our understanding what constitutes successful participation in these programs. We capitalize on the physics Olympiad where we have a comprehensive literature base delineating the characteristic features of expertise. This enabled us to utilize research-based instruments to measure expertise in the Physics Olympiad from a cognitive vantage point and triangulate these results with more motivational measures that have been proposed elsewhere. In order to assess characteristics for success in the Physics Olympiad participating students were tracked throughout their engagement in the Physics Olympiad with a questionnaire. 

\end{document}
